\documentclass[12pt, a4paper]{article}
\usepackage[english, russian]{babel}
\usepackage[T2A]{fontenc}
\usepackage[utf8]{inputenc}
\usepackage[left=3cm,right=1.5cm,top=2cm,bottom=2cm,bindingoffset=0mm]{geometry}
\usepackage{titlesec}
\usepackage{amsmath}
\usepackage{setspace}
\usepackage[pdftex]{graphicx}
\usepackage{dsfont}
\usepackage{comment}
\usepackage[unicode, pdftex]{hyperref}
\usepackage{fancyhdr}
\usepackage{colortbl}
\usepackage{indentfirst}

\graphicspath{{C:/Users/9/Desktop/market_mod}}
\setlength{\parindent}{1.25cm}
\linespread{1.5}


\begin{document}

\pagestyle{fancy}
\fancyhf{}
\renewcommand{\headrulewidth}{0pt}

\begin{center}
САНКТ-ПЕТЕРБУРГСКИЙ ГОСУДАРСТВЕННЫЙ УНИВЕРСИТЕТ\\
Направление: 01.03.02 Прикладная математика и информатика \\
ООП: Прикладная математика, фундаментальная информатика и программирование \\
Кафедра технологии программирования \\

\vspace*{2cm}
\large \textbf{ОТЧЕТ О НАУЧНО-ИССЛЕДОВАТЕЛЬСКОЙ РАБОТЕ}
\end{center}
\vspace*{1.5cm}

\begin{flushleft}
\textbf{Тема задания:}
\end{flushleft}
\hspace{1cm} Исследование вероятностной модели предсказания лучшей цены в биржевом \\ стакане по трейдовым данным
\vspace*{0.5cm}

\begin{flushleft}
\textbf{Выполнил:}
\end{flushleft}
\hspace{1cm} Смирнов Алексей Артёмович, группа 21.Б02-пу 
\vspace*{0.5cm}

\begin{flushleft}
\textbf{Руководитель научно--исследовательской работы:}
\end{flushleft}
\hspace{1cm} Блеканов Иван Станиславович, кандидат технических наук, доцент, \\ заведующий кафедрой технологии программирования
 
 
\vspace*{7cm}
\begin{center}
САНКТ-ПЕТЕРБУРГ \\
2024
\end{center}

\newpage

\renewcommand{\contentsname}{\begin{center}Содержание\end{center}}
\tableofcontents

\newpage

\fancyfoot[C]{\thepage}

\section{Введение}

Одна из основных проблем любого участника рынка --- неполная информация о состоянии самого рынка. Соответственно, любой участник рынка желает получить как можно больше информации, на основе которой и будет принимать дальнейшие действия, так как чем большим объемом информации он владеет, тем более правильным будет само действие с точки зрения желаемых результатов. Для увеличения количества информации о рынке используют модели, которые основаны на том, что рынок есть сильно взаимосвязная система, поэтому, имея некоторую часть информации о состоянии системы, можно делать предположения о состоянии неизвестных частей системы по определенным законам, которые могут быть как строго математическими, так и эмпирическими.

Данная работа посвящена изучению одной вероятностной модели о биржевом стакане (Orderbook), где под термином биржевого стакана подразумевается некоторая рыночная структура, на которой взаимодействуют через посредника в виде биржи две стороны --- покупатели и продавцы. Более подробное описание биржевого стакана будет дано в разделе $\hyperlink{orb}{\text{2 Микроструктура биржевого стакана}}$.

\subsection{Актуальность работы}
Изучаемая модель направлена на оценивание неизвестных, в определенный момент времени, параметров рынка, которые могут быть использованы для решения известной задачи Execution, которая является частью отрасли Financial Technologies, связанной с улучшением и оптимизацией финансовых услуг. Более формальное описание и обсуждение задачи Execution будет дано в разделе $\hyperlink{exec}{\text{3.1 Задача Execution}}$.

\subsection{Цели и задачи работы}
Основная цель работы --- оценить применимость предложенной модели, описать ситуации, при которых модель себя показывает лучше или хуже предполагаемого и, соответственно, определить практическую ценность данного статистического подхода, путем тестирования модели на реальных данных.

\newpage
\hypertarget{orb}{
\section{Микроструктура биржевого стакана}}

Биржа или биржевой стакан представляет собой две стороны рынка --- продавцов (ask сторона) и покупателей (bid сторона). Каждая сторона состоит из множества ценовых уровней (price levels), где каждый ценовой уровень является очередью заявок (orders) с фиксированными объемами (на ask стороне объем на продажу, на bid стороне объем желаемой покупки), порядок очереди определяется порядком появления заявок на бирже. Соответственно, лучшая цена на ask стороне называется лучшей ценой покупки, а лучшая цена на bid стороне, аналогично, называется лучшей ценой продажи.

\hypertarget{Orderbook}{
\begin{figure}[htbp]
\center{\includegraphics[width=1\textwidth]{Orderbook.png} Рисунок 1 --- Микроструктура биржевого стакана}
\end{figure}}

Как показано на $\hyperlink{Orderbook}{\text{Рисунке 1}}$, значения $p_i$ --- ценовые уровни, которые разбиты 
по объемам отдельных участников рынка, соответственно, общий объем, продающийся на конкретном уровне, есть сумма всех объемов очереди данного уровня.

\subsection{Основные характеристики биржевого стакана}

Величина $p_1 - p_{-1}$ есть величина спреда (spread), численно равный разности лучшей цены ask стороны и лучшей цены bid стороны, но имеющий прямое отношение к волатильности продукта, а именно, чем он больше, тем менее волатилен продукт. Это объясняется тем, что современная рыночная теория предполагает, что в каждый конкретный момент продукт имеет эталонную цену $(p_{ref})$, то есть истинную цену продукта в данный момент времени, купив или продав продукт по этой цене участник рынка считает, что он ничего не потерял и ничего не приобрел, следовательно желание продавцов --- продать продукт по большей цене, чем та, которую они считают для себя $p_{ref}$, а желание покупателей --- купить продукт дешевле, чем их субъективное представление о значении $p_{ref}$. Получаем, что значение $p_{ref}$ должно лежать в промежутке между лучшими ценами в стабильном состоянии биржевого стакана, так как иначе будет существовать такой ценовой уровень, по которому участники рынка будут считать выгодным для себя заключить сделку и, следовательно, этот ценовой уровень исчезнет. Очевидно, что ни один участник рынка точно не может знать значение $p_{ref}$, но каждый знает, что это значение находится внутри спреда и каждый участник рынка имеет ожидания по его движению в сторону увеличения или уменьшения, поэтому, если продукт является высоковолатильным, то частота сделок позволяет всем участникам рынка уточнить значение $p_{ref}$ и иметь более точные представления о его движении на коротких промежутках времени. Это означает, что интервал возможных значений $p_{ref}$ уменьшается для всех участников рынка, из-за чего ценовые уровни могут существовать довольно близко к истинному значению $p_{ref}$, так как не находится достаточное количество участников рынка, для которых данный уровень принадлежит их ожидаемому интервалу возможных значений $p_{ref}$. Иными словами, спред есть ожидаемый интервал возможных значений $p_{ref}$, согласованный всеми участниками рынка и ,как было показано выше, для высоковолатильных продуктов характерно иметь меньшую величину спреда, чем для низковолатильных, у которых участники рынка имеют меньше информации о значении $p_{ref}$ и, соответственно, ожидаемый интервал возможных значений $p_{ref}$ больше, то есть величина спреда больше.

Также стоит упомянуть, что на практике все численные значения на бирже принимают дискретные значения и величина между соседними значениями есть тик (tick). Например, для вычисления среднего значения цены продукта $(p_{mid})$, которую часто использую для оценки $p_{ref}$, применяют следующую формулу $\hyperlink{qrm}{[1]}$:

\[
\left\{
\begin{array}{cc}
\displaystyle \frac{p_1 + p_{-1}}{2} \quad \quad & if \quad \displaystyle\frac{p_1 + p_{-1}}{2} \notin \mathds{Z} \cdot t, \\
\displaystyle \frac{p_1 + p_{-1}}{2} \pm \frac{t}{2} \quad \quad & if \quad\displaystyle \frac{p_1 + p_{-1}}{2} \in \mathds{Z} \cdot t
\end{array}\right.,
\eqno(1)
\hypertarget{1}
\]
где $t$ - величина тика ценовых уровней. Суть формулы $\hyperlink{1}{(1)}$ в том, что значение $p_{ref}$ должно быть недостижимым значением ценового уровня для участников рынка, так как в противном случае все сделки происходили бы на этом уровне.

\subsection{Взаимодействие участников рынка на биржевом стакане}

Участники биржевого стакана имеют несколько инструментов взаимодействия между собой, а именно, любой участник может отправить три вида заявок на биржу: CancelOrder, LimitOrder, MarketOrder. 

CancelOrder или заявка на отмену --- может отправить участник, у которого есть незакрытые заявки в биржевом стакане. Участник рынка посылает бирже данную заявку, указывая ценовой уровень и объем, после чего биржа, подтверждая валидность операции, производит снятие заявки.

MarketOrder или заявка на продажу/покупку --- может отправить участник, который хочет в кратчайшие сроки продать/приобрести товара фиксированного объема. Биржа, при получении MarketOrder, пытается удовлетворить желание участника по минимально возможной цене, то есть производит продажу/покупку по лучшим ценам на стороне, в случае, если объем заявки превышает объем лучшего уровня, то производятся сделки на следующем ценовом уровне, который считается новым лучшим уровнем. При проведении сделок создается трейд (trade), в котором указан ценовой уровень, объем и стороны сделки, которыми являются составитель заявки и участники стакана, чьи заявки были полностью или частично выполнены. На одном ценовом уровне трейды создаются с участниками в порядке их очереди на биржевом стакане.

LimitOrder или ограниченная заявка --- может отправить участник, указав продажу/покупку, ценовой уровень, объем и тип заявки. Стандартными\footnote[1]{На практике биржа может использовать дополнительные типы заявок, но данные типы считаются самыми распространенными, которые будут учитываться в данной работе.} типами заявок считаются: fill--or--kill(FOK), immediate--or--cancel(IOC), good--till--cancel(GTC). 

Если LimitOrder имеет тип GTC, то биржа, получив данную заявку, попытается ее немедленно полностью или частично выполнить, то есть, если на данном ценовом уровне или ниже присутствуют заявки противоположной стороны суммарного объема не меньшего, чем указано в заявке, то заявка тут же исполняется, создаются трейды по алгоритму описанному выше, если же суммарного объема недостаточно, то заявка исполняется частично, на весь подходящий объем, при этом, биржа гарантирует, что цена всех сделок не выше, чем та, которую указал составитель в заявке, а оставшийся невыполненный объем заявки помещается в биржевой стакан на указанный ценовой уровень в конец очереди. Тип GTC выбирают те участники, которые не против того, что их заявка окажется в биржевом стакане, пока она не выполнится. 

Если LimitOrder имеет тип IOC, то происходят все вышеописанные действия для типа GTC, но, если возникает ситуация, когда заявка выполнилась на максимально возможный объем по цене, не превышающей указанную в заявке, то весь остальной объем возвращается участнику, не вставая в очередь и, соответственно, в биржевой стакан. Тип IOC выбирают те участники, которые согласны на частичное выполнение заявки в данный момент времени, но не хотят участвовать в биржевом стакане.

Если LimitOrder имеет тип FOK, то происходят все те же действия, но если возникает ситуация, что заявка не может быть исполнена в данный момент времени, то ее выполнение полностью отменяется. Тип FOK выбирают те участники, которых удовлетворяет только полное выполнение заявки в данный момент времени без участия в биржевом стакане.

Как видно, CancelOrder никогда не создают трейды, MarketOrder гарантированно создают трейды, а LimitOrder лишь в определенных ситуациях, что будет крайне важно для дальнейших исследований модели, а также то, что трейды всегда происходят по лучшим ценам.

\subsection{Практические особенности получения данных с бирж}

Как было сказано ранее, каждый участник рынка желает получить полную информацию о самом рынке, но данное желание не может быть реализовано даже технически, так как, когда на бирже произошло какое-то событие, которое она транслирует через известные участникам информационные каналы, из-за задержек каналов у участников нет никаких гарантий, что за время прохождения информации по каналу, внутри биржи не было обработано другое событие и их представление о состоянии рынка является актуальным. Одновременно с этим, задержка получения информации у разных участников отличается, из-за чего в определенные моменты нарушается симметрия информации участников рынка, что противоречит постулатам идеального рынка, на которых основан непрерывный аукцион двойной цены применяемый большинством бирж.

Для того, чтобы избежать малейших рассинхронизаций внутри биржи, все изменения состояния биржи происходят линейно, то есть однопоточно, по внутреннему времени этого потока и определяется точное время любых изменений.

Рассмотрим пример работы с биржей на примере криптовалютной биржи \\ BINANCE, а именно через предоставляемый самой биржей BINANCE API $\hyperlink{binance_api}{[2]}$. 

BINANCE API имеет свой протокол REST API для получения различной информации, но в данной работе рассмотрим только Market Data endpoints, который возвращает снепшот (sneapshot), то есть полное состояния биржевого стакана конкретного продукта. У данного протокола есть ряд ограничений, одно из которых --- частота отправки запроса, которая является хоть и динамической, то есть биржа ответом на запрос также возвращает временной промежуток, который требуется выждать, чтобы выслать следующий запрос, но в рамках частоты торгов на данной криптобирже является очень большим ограничением, так как среднее время между запросами составляет несколько минут, то есть только по данной информации невозможно следить за состоянием биржевого стакана в реальном времени и предпринимать какие-либо действия. Еще одним недостатком является отсутствие у снепшота времени отправки (vanue tamestamp), то есть по ответу биржи невозможно определить задержку канала связи, но у снепшота есть параметр --- конечный id события, которое было учтено в данном снепшоте, соответственно, если получать все события происходящие на бирже, то сравнивая их id и конечный id снепшота, то можно восстановить момент в который был отправлен снепшот, так как биржа гарантирует последовательное инкрементальное увеличение id всех событий происходящих на бирже в силу однопоточности ядра.

У BINANCE API есть соответствующие Web Socket Streams (WSS) различных типов по каждому продукту. В данной работе рассмотрим два типа: Trade Stream и Diff. Depth Stream. Когда на бирже происходит трейд, то информация сразу же транслируется по Trade Stream с соответствующей информацией о произошедшей сделке, в том числе время трейда, которое может не совпадать со временем обработки события, которое породило данный трейд, так как формирование трейдов происходит параллельно с формированием иной информации о произошедшем событии, следовательно, время трейда есть внутреннее время данного потока, которое может отличаться от времени любого другого потока. Транслирование событий (Depth), в которых записаны все изменения биржевого стакана с момента последнего Depth, в Diff. Depth Stream происходит по времени, можно получать событие каждые 100 или 1000 миллисекунд, что также довольно медленно для высокочастотной торговли. Благодаря одному снепшоту из Market Data endpoints и постоянному получению Depth из Diff. Depth Stream и накладыванию этих изменений на локальный биржевой стакан, можно получать актуальное состояние биржевого стакана каждые 100 миллисекунд. 

Аналогично, можно накладывать на локальный биржевой стакан и все трейды из Trade Stream, хоть это не даст полного актуального состояния из-за того, что нетрейдовые события будут учтены только в Depth, но даст возможность поддерживать некоторую актуальность лучших цен, хотя не гарантированную. 

Стоит отметить, что в снепшотах, Depth и трейдах объемом обозначается общий объем ценового уровня, без разделения по участникам, то есть, даже если теоретически поддерживать актуальный биржевой стакан в любой момент времени по данным событиям, то у нас все равно не будет полного состояния биржевого стакана, так как не будет никакой информации об очередях на ценовых уровнях.

Теоретически может возникнуть проблема с синхронизацией событий с разных WSS, которые формируют разные потоки биржи, то есть возможна ситуация, когда события с разных потоков придут не в том порядке, в котором они были отправлены биржей, а время отправления данных событий не может быть сравнено в силу отсутствия гарантий синхронизаций времени потоков, на этот случай можно полагаться на id событий, например Depth хранит два значения id --- начальный и конечный id учтенных событий, по которым можно восстановить его относительное расположение относительно других событий имеющих id из других WSS, но у BINANCE API существуют WSS, которые невозможно синхронизировать с другими, например, Individual Symbol Book Ticker Streams, в котором отправляются события, хранящие лучшие цены каждой стороны, а также его относительное расположение относительно других событий с данного WSS, следовательно, у нас никакой информации для синхронизации событий с данного WSS и любого другого WSS.

\section{Модель}

\hypertarget{exec}{
\subsection{Задача Execution}}

Известная задача Execution может формулироваться в нескольких вариациях, например, пусть в данный момент времени пришел сигнал, после которого нужно купить максимальное количество товара на фиксированное количество денег за фиксированное время, или, пусть в данный момент времени пришел сигнал, после которого нужно купить фиксированное количество товара за минимальное количество денег за фиксированное время. В данной работе под задачей Execution будет подразумеваться вторая формулировка задачи.

Как видно из формулировки, исполняющий заявку на Execution должен быть готов выполнить в любое время, когда пришел сигнал, то есть нужно иметь возможность выполнять данную заявку даже не имея полной информации о состоянии рынка, при этом время на исполнение заявки может быть недостаточно, чтобы дождаться ситуации, когда исполняющий сможет получить хоть какую-то дополнительную информацию о состоянии рынка, поэтому данная задача требует выполнить заявку с информацией в момент сигнала, а, так как сигнал может прийти в любой момент времени, то, следовательно, требуется знать необходимые параметры состояния рынка для принятия решений в любой момент времени, именно для предсказания одного из таких параметров и предлагается вероятностная модель, которая будет описана ниже.

\subsection{Построение модели}
 
Предлагаемая модель основывается на некоторых эмпирических вероятностных распределениях определенных параметров биржевого стакана, а именно:

\begin{enumerate}
\item Время между трейдами распределено по экспоненциальному закону распределения, то есть мы считаем, что участники рынка торгуют и не торгуют в одни и те же моменты времени, можно сказать, что происходят всплески активности трейдов, которые образуют некоторые группы, разделенные меньшей активностью ($\hyperlink{trade_exp}{\text{Рисунок 2}}$).

\item Абсолютное отклонение цены в биржевых событиях от лучшей цены на соответствующем уровне распределено по Пуассоновскому распределению, что естественно предложить, ведь все трейды происходят на лучших ценах, соответственно, абсолютное смещение лучшей цены распределено по Пуассоновскому распределению, которое является дискретным, что хорошо описывает практическую сторону реализации хранения чисел биржей, как было показано в конце параграфа 2.1.
\end{enumerate}

Модель будет предсказывать смещение лучшей цены одной стороны по данным полученным по трейдам, то есть по ряду лучших цен, ряду времени между трейдами и прошлой известной лучшей цене, то есть без учета объемов трейдов, путем построения доверительной области (в нашем случае двумерной) с учетом корреляции признаков, ведь как говорилось ранее --- биржевой стакан сильносвязная система, поэтому именно благодаря корреляции признаков модель будет учитывать время получения трейдов при определении доверительного интервала для смещения лучшей цены. 
\newpage
Для определения многомерной доверительной области среднего значения зависимых нормально распределенных случайных величин используют обобщенную статистику Стьюдента $t^2$ --- статистику Хотеллинга $T^2$ $\hyperlink{hotteling}{[3]}$ ($\hyperlink{inter}{\text{Рисунок 3}}$).
	
\begin{figure}[h]
\begin{minipage}[h]{0.5\linewidth}
\center{\includegraphics[width=1\linewidth]{trade_exp.png} \\ \hypertarget{trade_exp}{Рисунок 2 --- Симулированное распределение трейдов}}
\end{minipage}
\hfill
\begin{minipage}[h]{0.5\linewidth}
\center{\includegraphics[width=1\linewidth]{inter.png} \\ \hypertarget{inter}{Рисунок 3 --- Уменьшение условного доверительного интервала при использовании учета корреляции признаков}}
\end{minipage}
\label{ris:image1}
\end{figure}

Пусть имеется $n$ измерений каждой из $k$ одномерной нормально распределенной случайной величины, которые могут быть зависимы между собой, тогда доверительная область с вероятностью $p$ для $k$-мерного вектора средних значений $\mu$ есть $k$-мерный эллипсоид, который задается следующим выражением:
\[
T^2 \geq n (\overline{X} - \mu)^TC^{-1}(\overline{X} - \mu),
\eqno(2)
\]
где $C$ --- матрица $k \times k$ выборочных оценок ковариаций, $\overline{X}$ --- выборочная оценка вектора средних значений, $T^2$ --- статистика Хотеллинга, равная $\displaystyle \frac{k(n-1)}{n-k}F^p_{v_1, v_2}$, где $F^p_{v_1, v_2}$ --- квантиль распределения Фишера с вероятностью $p$ и степенями свободы $v_1 = k, v_2 = n - k$. Так как нас интересуют только доверительный интервал для одной компоненты (Н.У.О для компоненты $\mu_k$) вектора средних значений, то мы полагаем все остальные компоненты равные выборочным оценкам $\mu_i = \overline{X}_i \;\forall i = \overline{1,k-1}$, то есть сечем $k$-мерный эллипсоид одномерной прямой, проходящей через его центр и параллельной оси $k$-той компоненты. 


Соответственно получаем:
\[
\begin{array}{c}
\displaystyle \frac{k(n-1)}{n(n-k)}F^p_{k, n-k} \geq c_{kk}(\overline{X_k} - \mu_k)^2 \Rightarrow \\
\\
\Rightarrow \displaystyle \overline{X_k} - \sqrt{\frac{1}{c_{kk}}\frac{k(n-1)}{n(n-k)}F^p_{k, n-k}} \leq \mu_k \leq \overline{X_k} + \sqrt{\frac{1}{c_{kk}}\frac{k(n-1)}{n(n-k)}F^p_{k, n-k}} \:,
\end{array}
\hypertarget{3}
\eqno(3)
\]
где $c_{kk}$ --- элемент $k$-той строки и $k$-того столбца обратной матрицы выборочных оценок ковариации. Для случая $k=2$, который будет представлен в данной работе, значение $c_{22}$ определяется как: 
\[
\displaystyle c_{22} = \frac{cov(\xi_1, \xi_1)}{cov(\xi_1, \xi_1)cov(\xi_2, \xi_2) - cov(\xi_1, \xi_2)^2},
\eqno(4)
\] 
где $cov$ --- значение выборочной ковариации, $\xi_1$ --- генеральная совокупность для значений интервалов времени между трейдами, $\xi_2$ --- генеральная совокупность для значений смещения лучшей цены в трейдах.

По формуле $\hyperlink{3}{(3)}$ можно рассчитать доверительный интервал для среднего значения нормально распределенной величины, но в начале данного параграфа были выдвинуты предположения о распределении исследуемых величин, которые не являются нормальными, поэтому требуется провести нормализацию распределений, путем преобразования рассматриваемых значений.

Стандартным путем нормализации экспоненциально распределенной величины является ее логарифмирование, то есть вместо значений $\xi_1$ будут использоваться моделью значения $\ln(\xi_1)$, а для нормализации величины $\xi_2$, распределенной по Пуассоновскому распределению, часто берут ее корень, то есть $\sqrt{\xi_2}$, но, так как смещение лучшей цены может быть в обе стороны, то стоит учитывать знак значения. Соответственно, при получении трейда, который имеет ценовой уровень $p_t$ и интервал времени между ним и прошлым событием $t$, и известной информации о прошлой лучшей цене $p_{best}$ соответствующей стороны, будут формироваться значения $\ln(t), \text{sign}(p_t - p_{best})\sqrt{|p_t - p_{best}|}$, которые будем считать нормально распределенными зависимыми случайными значениями.

Из неравенства $\hyperlink{3}{(3)}$:
\[
\displaystyle \overline{X_2} - \sqrt{\frac{1}{c_{22}}\frac{k(n-1)}{n(n-2)}F^p_{k, n-2}} \leq \text{sign}(\eta_2 - p_{best})\sqrt{|\eta_2 - p_{best}|} \leq \overline{X_2} + \sqrt{\frac{1}{c_{22}}\frac{k(n-1)}{n(n-2)}F^p_{k, n-2}} \:,
\hypertarget{5}
\eqno(5)
\]
где $\eta_2$ --- среднее значение ценового уровня лучшей цены оцениваемой стороны.

Нетрудно показать, что:
\[
\displaystyle a \leq b \;\Leftrightarrow\; \text{sign}(a) \cdot a^2 \leq \text{sign}(b) \cdot b^2.
\hypertarget{6}
\eqno(6)
\]
Применяя дважды $\hyperlink{6}{(6)}$ к $\hyperlink{5}{(5)}$, получим:

\[
\begin{array}{c}
\displaystyle \text{sign}\left(\overline{X_2} - \sqrt{\frac{1}{c_{22}}\frac{k(n-1)}{n(n-2)}F^p_{k, n-2}}\right) \left(\overline{X_2} - \sqrt{\frac{1}{c_{22}}\frac{k(n-1)}{n(n-2)}F^p_{k, n-2}}\right)^2 \leq \\
\displaystyle \leq \text{sign}(\eta_2 - p_{best})|\eta_2 - p_{best}| \leq \\
\displaystyle \leq \text{sign}\left(\overline{X_2} + \sqrt{\frac{1}{c_{22}}\frac{k(n-1)}{n(n-2)}F^p_{k, n-2}}\right) \left(\overline{X_2} + \sqrt{\frac{1}{c_{22}}\frac{k(n-1)}{n(n-2)}F^p_{k, n-2}}\right)^2,
\end{array}
\eqno(7)
\]
и, заменяя $\text{sign}(\eta_2 - p_{best})|\eta_2 - p_{best}| = \eta_2 - p_{best}$, получаем оценку доверительного интервала для среднего значения лучшей цены $\eta_2$:

\[
\begin{array}{c}
\displaystyle p_{best} + \text{sign}\left(\overline{X_2} - \sqrt{\frac{1}{c_{22}}\frac{k(n-1)}{n(n-2)}F^p_{k, n-2}}\right) \left(\overline{X_2} - \sqrt{\frac{1}{c_{22}}\frac{k(n-1)}{n(n-2)}F^p_{k, n-2}}\right)^2 \leq \\
\displaystyle \leq \eta_2 \leq \\
\displaystyle \leq p_{best} + \text{sign}\left(\overline{X_2} + \sqrt{\frac{1}{c_{22}}\frac{k(n-1)}{n(n-2)}F^p_{k, n-2}}\right) \left(\overline{X_2} + \sqrt{\frac{1}{c_{22}}\frac{k(n-1)}{n(n-2)}F^p_{k, n-2}}\right)^2.
\end{array}
\hypertarget{8}
\eqno(8)
\]

Данная оценка является смещенной из-за преобразований данных к нормально распределенным, поэтому предлагается сместить весь промежуток так, чтобы его центральной точкой была последняя известная лучшая цена, которая равна $p_{best}$, в случае отсутствия трейдов за наблюдаемый промежуток, либо цене последнего трейда.

Будем применять данную интервальную оценку если произошли хотя бы четыре трейда на интересующей нас стороне, что делает ее применимой только для товаров имеющих уровень волатильности больше, чем некоторый минимально требуемый, но если, как в случае BINANCE, за 100 миллисекунд не произошло достаточное количество трейдов, то можно считать, что лучшая цена товара не изменилась, хотя, конечно, возможно, что на рынке данного товара происходит огромное количество нетрейдовых событий, которые могли изменить лучшую цену на сколь угодно большое значение, но в данной работе считаем данный сценарий маловероятным, и в случае недостаточного количества трейдов будем считать, что модель предсказывает, что интервальная оценка вырождается в точечную, где среднее значение в случае отсутствия трейдов $\eta_2 = p_{best}$, а в случае существования хотя бы одного трейда $\eta_2 = p_t$, где $p_t$ --- цена последнего трейда. Дополнив формулу $\hyperlink{8}{(8)}$ точеченой оценкой в случае малого количества трейдов получаем, что данная модель позволяет иметь оценку для лучшего ценового уровня в любой момент времени по трейдовым данным.

Из рассуждений выше об ограничениях применения интервальной оценки модели сразу видны ее недостатки, а именно то, что учитываются только трейдовые события, а вклад нетрейдовых мы никак не можем даже оценить, но мы можем лишь надеяться на то, что при большом количестве нетрейдовых событий происходит и большое количество трейдовых, так как биржа сильносвязная система, благодаря которым мы может поддерживать актуальную оценку.

Насколько критичны недостатки модели и насколько адекватна реальности сама оценка лучше всего понять при  использовании данной модели на практике и сравнении истинного значения с предсказанными.

\section{Практические результаты}

\subsection{Вероятностный симулятор взаимодействия участников \\ биржевого стакана}

Для проверки модели был построен симулятор биржевого стакана, где происходила полная искусственная вероятностная генерация всех взаимодействий участников биржи. В данном симуляторе генерируются все типы заявок: CancelOrder, MarketOrder, LimitOrder.

Симулируемый биржевой стакан имеет две стороны --- ask и bid стороны, каждая сторона симулируется на $a_1$ ценовых уровнях, которые хранятся в виде смещений относительно своих лучших цен (от $0$ до $a_1 - 1$).  

MarketOrder имеет два параметра --- сторону стакана, на которой происходит сделка, и объем сделки. Сторона сделки генерируется путем получения равномерно распределенной случайной величины на $(0, 1)$ и сравнения данного значения с некоторым пороговым значением $m_1$, а объем MarketOrder генерируем как нормально распределенную величину со средним значением $m_2$ и среднеквадратичным отклонением $m_3$. 

CancelOrder имеет три параметра --- сторону стакана, ценовой уровень и объем. Сторона стакана генерируется одновременно со смещением ценового уровнем $b$, который определяется как:
\[
b = \left\{
\begin{array}{c}
a_1 - \xi_{a_1} - 1 \quad if \; \xi - a_1 < 0, \\
\xi_{a_1} \quad \quad \quad \quad \; \; if \; \xi - a_1 \geq 0
\end{array}\right.,
\hypertarget{9}
\eqno(9)
\]
где $\xi \sim Bin(2a_1 - 1, c_1)$, $\xi \equiv \xi_{a_1} \mod a_1$, то есть значение смещения цены определяется случайной величиной, распределенной по биномиальному распределению с варьируемым параметром вероятности $c_1$. Если реализовался первый случай формулы $\hyperlink{9}{(9)}$, то CancelOrder произошел на bid стороне, иначе на ask стороне. Объем CancelOrder генерируем также, как и объем MarketOrder, но со средним значением $c_2$ и среднеквадратичным отклонением $c_3$.

LimitOrder имеет пять параметров --- сторону стакана, ценовой уровень, объем, тип заявки, сторона заявителя. Сторона стакана, ценовой уровень и объем генерируются аналогично CancelOrder, где в $\hyperlink{9}{(9)}$ $\xi \sim Bin(2a_1 - 1, l_1)$, среднее значение объема есть $l_2$ и среднеквадратичное отклонение $l_3$. Тип заявки определяется случайным равномерным распределением на $(0, 1)$, где тип FOK получается с вероятностью $l_4$, IOC с вероятностью $l_5$, GTC с вероятностью $1 - l_4 - l_5$. Сторона заявителя не совпадает со стороной заявки с вероятностью $l_6$.

Может возникнуть вопрос, почему для генерации смещения цены было выбрано биномиальное распределение, а не Пуассоновское, которое используется в модели? Это было сделано для того, чтобы протестировать модель в ситуации, когда наши предположения о распределении неверны, но общие статистические законы, что большинство CancelOrder и LimitOrder происходят около лучших цен, остаются верными, а также то, что тестировать модель, в которую заложены определенные распределения, на этих же распределениях не имеет большого смысла, так как на практике полного совпадения ожидаемых и реальных распределений никогда не произойдет.

При генерации событий MarketOrder появляется с вероятностью $p_M$, LimitOrder с вероятностью $p_L$, CancelOrder с вероятностью $1 - p_M - p_L$, а присваивамое время событию отличается от времени предыдущего на величину $\tau \sim Exp(a_2)$, то есть распределение событий по времени совпадает с тем, что предполагается в модели.

По итогу, симулятор имеет 16 начальных параметров: 
\[
\begin{array}{c}
a_1, a_2, p_M, p_L \text{--- параметры симулирования биржевого стакана}, \\
m_1, m_2, m_3 \text{--- параметры симулирования MarketOrder}, \\
c_1, c_2, c_3 \text{--- параметры симулирования CancelOrder}, \\
l_1, l_2, l_3, l_4, l_5, l_6 \text{--- параметры симулирования LimitOrder},
\end{array}
\]
варьируя которые, можно получить различные поведения биржевых торгов.

\subsection{Результаты тестирования модели на симуляторе биржевого стакана}

Основной интересующей характеристикой модели приведенной выше есть точность предсказания, требуется понять при каких параметрах торгуемого товара лучшая цена действительно будет принадлежать доверительному интервалу с вероятностью $p$, предсказанному моделью, с той же вероятностью. Основной варьируемой характеристикой товара будет являться $p_T$ --- доля трейдовых событий от общего числа событий, которая зависит от параметров симулятора как:
\[
p_T = p_M + p_L l_6.
\eqno(10)
\hypertarget{10}
\]
Формула $\hyperlink{10}{(10)}$ на самом деле является оценкой сверху доли трейдовых событий, так как не любая трейдовая заявка LimitOrder может быть выполнена, но при достаточно большом объеме товара на лучшем уровне трейдовая заявка LimitOrder будет обязательно выполнена, поэтому невыполнение данной заявки может быть только в переходный момент лучшей цены, доля которых от общего числа торгов крайне мала.

Для тестирования модели были сгенерированы 20 вариантов поведения товаров, содержащих 10 млн событий каждый, со значениями $p_T = 0.01, 0.02, \dots, 0.19, 0.20$, то есть количество трейдовых событий варьируется от $1\%$ до $20\%$, а единственный параметр модели выбран равным $p = 0.95$. Все точные значения шестнадцати параметров, использованные для генерации данных приведены в $\hyperlink{table_sim}{\text{Таблице А.1}}$, где $i$ --- соответствующее значения $p_T$ в процентах. Данные собираются в течении ста симулируемых миллисекунд, по истечении которых, все произошедшие трейды используются моделью для одного предсказания, после чего собираются данные следующие сто симулируемых миллисекунд и так далее. Для тестирования разделим все данные на два типа: данные, где количество трейдов меньше четырех и не менее четырех, так как в первом случае мы получаем вырожденную оценку.

\hypertarget{Sim_res_1}{
\begin{figure}[h]
\center{\includegraphics[width=1\textwidth]{Sim_res_1.png} Рисунок 4 --- Результаты тестирования модели на симулируемых данных}
\end{figure}}

На $\hyperlink{Sim_res_1}{\text{Рисунке 4}}$ представлены результаты тестирования модели. Как и ожидалось, доля верных предсказаний (верным предсказанием считается тот случай, когда истинная лучшая цена попадает в доверительный интервал) растет с увеличением доли трейдовых событий, на которых и происходит предсказание, при этом доля правильных предсказаний для данных с количеством трейдов не менее четырех сильно лучше при малой доли трейдовых событий, но разница доли правильных ответов сокращается по мере увеличения доли трейдовых событий. 

Также стоит отметить, что при малой доли трейдовых событий на данных с количеством трейдов менее четырех происходит провал ниже ожидаемого значения $p$, а на данных с количеством трейдов не менее четырех модель везде достигает необходимой планки. 

На $\hyperlink{Sim_res_1}{\text{Рисунке 4}}$ также видно то, как резко падает доля верных ответов при уменьшении доли трейдовых событий, то есть для применения модели требуется некоторая минимальная доля трейдовых событий, а в случае, если этот порог не достигнут, то модель крайне быстро теряет предсказательную силу.

Доля данных с количеством трейдов не менее четырех составляет всего $1.7\%$ от общего количества. Данное значение было выбрано таковым, чтобы симуляция была приближена к реальным значениям, именно поэтому, при изменении доли трейдовых событий, мы также изменяем интенсивность $a_2$ генерации событий по времени.

Стоит отметить, что предсказание модели на данных количеством трейдов менее четырех является точечной оценкой, а на данных с количеством трейдов не менее четырех является интервальной, поэтому стоит посмотреть на значения длин интервальных оценок, которые представлены на $\hyperlink{Sim_res_2}{\text{Рисунке 5}}$.

\hypertarget{Sim_res_2}{
\begin{figure}[h]
\center{\includegraphics[width=1\textwidth]{Sim_res_2.png} Рисунок 5 --- Средняя длина интервальной оценки на данных с количеством трейдов не менее четырех}
\end{figure}}

Видим, что средняя длина интервальной оценки также уменьшается с ростом доли трейдовых событий и, при доли в $2\%$ и более, имеет значение менее одного тика, то есть в среднем однозначно определяет значение лучшей цены, так как значения ценовых уровней дискретны и кратны тикам.

По итогу, на симулируемых данных мы увидели, что для применения модели существует необходимый порог доли трейдовых событий, который примерно равен $2\%$, после которого модель достигает необходимой доли правильных предсказаний, даже с учетом того, что в симуляции использовалось иное распределение событий по ценовым уровням, чем то, которое закладывалось при конструировании модели, при этом по интервальной оценке можно в среднем однозначно восстановить лучшую цену, но, в случае, если доли трейдовых событий недостаточно, то модель крайне быстро теряет предсказательную силу --- быстро уменьшается доля корректных ответов и увеличивается средняя длина интервальной оценки.

\subsection{Результаты тестирования модели на реальных данных}

Для тестирования модели на реальных данных были использованы записи пятнадцати валютных пар с биржи BINANCE в течении одной недели непрерывных торгов. Критерием выбора валютной пары для наблюдений была волатильность, соответственно, все валютные пары, где одной из валют является USDT, были отсортированы по валатильности, после чего были взяты 5 высоковолатильных валют, 5 средневолатильных, 5 низковолатильных. Аналогично, как и в симулируемых данных, каждые сто миллисекунд собиралась информация о трейдах, то есть информация о всех трейдах между получением двух Depth, после чего модель делала предсказание на основе данной информации, а информацию о истинном значении получали из последующего Depth, так как это единственный момент, когда можно получить истинное значение лучшей цены, которое можно сравнить с предсказанием модели.

На $\hyperlink{Real_res_1}{\text{Рисунке 6}}$ показана доля верных предсказаний для каждой валютной пары по убыванию волатильности. Видим, что ни в одном случае не удалось достигнуть доли в $95\%$, но хорошо видно, что, как и в случае с симулированными данными, доля верных ответов на данных с количеством трейдов не менее четырех для некоторых валютных пар сильно больше, но никакой зависимости от волатильности товара не наблюдается. 

Долю данных с количеством трейдов не менее четырех для каждой валютной пары можно увидеть в $\hyperlink{table_real}{\text{Таблице B.1}}$, там же можно увидеть насколько мало происходит измерений, в которых имеется не менее четырех трейдов, из-за чего для низковолатильных валютных пар и большинства средневолатильных нельзя сделать вообще каких-либо выводов для таких измерений.

Но все же, столь низкий уровень доли правильных ответов можно объяснить двумя причинами, а именно то, что возможно доля трейдовых событий для данных валютных пар ниже, чем $1$--$2\%$, а также то, что механизм распределения событий по ценовым уровням гораздо сложнее, чем распределения, которые используются в модели, хотя, как было показано, что даже если товар имеет иное распределение, чем то, которое заложено в модели, предсказания модели все равно имеют смысл. 

\hypertarget{Real_res_1}{
\begin{figure}[h]
\center{\includegraphics[width=1\textwidth]{Real_res_1.png} Рисунок 6 --- Результаты тестирования модели на реальных данных}
\end{figure}}

На $\hyperlink{Real_res_2}{\text{Рисунке 7}}$ показана средняя относительная длина интервала к лучшей цене товара в процентах. Как видим, величина длины интервала для всех валютных пар не превышает $0.01\%$, то есть величина интервала довольно точно локализирует окрестность нахождения лучшей цены, но какой-либо корреляции данного параметра от волатильности продукта не наблюдается, но наблюдается некоторая зависимость доли верных ответов и средней относительной длины интервала к лучшей цене для высоковолатильных валютных пар.

Каждая валютная пара имеет свои характеристики поведения, поэтому с каждой валютной парой модель отрабатывает по разному, но видно, что для тех высоковолатильных валютных пар, для которых средняя относительная длина интервала к лучшей цене оказывается выше, доля верных ответов выше, что довольно логично, а для низковолатильных валютных пар и средневолатильных такой зависимости не наблюдается, кроме STRAXUSDT, но в силу малого числа данных, велика вероятность статистической погрешности.

\hypertarget{Real_res_2}{
\begin{figure}[h]
\center{\includegraphics[width=1\textwidth]{Real_res_2.png} Рисунок 7 --- Средняя относительная длина интервала к лучшей цене товара в процентах}
\end{figure}}

На мой взгляд, основная причина низкой доли верных предсказаний в том, что распределение событий по ценовым уровням реальных товаров не описываются стандартными, а тем более стационарными, распределениями, но модель все равно показала лучшие результаты при интервальной оценке, имеющей относительно малую длину, чем точечной, при условии достаточного количества трейдов в наблюдениях.

\section{Заключение}

\subsection{Результаты работы}

В данной работе была исследована вероятностная модель предсказания лучшей цены в биржевом стакане по трейдовым данным. Модель была протестирована на симулируемых данных и реальных данных, изучены недостатки и преимущества данного вероятностного подхода к прогнозированию лучшей цены. При тестировании модели на реальных данных было замечено большое отличие в результатах от полученных при тестировании на симулируемых, были даны возможные объяснения данному факту. Были явно описаны случаи применимости модели и последствия невыполнения данных условий.

\section{Список использованных источников}

\hypertarget{qrm} $[1]$ Simulating and analyzing order book data:
The queue-reactive model | Weibing Huang, Charles-Albert Lehalle and Mathieu Rosenbaum

\hypertarget{binance_api} $[2]$ \href{https://developers.binance.com/docs/binance-spot-api-docs/README}{\textcolor{blue}{BINANCE API}}

\hypertarget{hotteling} $[3]$ Applied Multivariate Statistical Analysis by Johnson and Wichern

\hypertarget{hotteling} $[4]$ The Econometrics of Financial Markets | John Y.Campbell, Andrew W. Lo, A. Craig MackKinlay

\newpage

\section{Приложения}

\subsection*{\begin{center}Приложение А. Параметры симулятора\end{center}}
\[
\begin{array}{|c|c|c|c|}
\hline
a_1 = 21 & a_2 = i/2 & p_M = (i - 1)/100 + 0.002 & p_L = 0.8 \\
\hline
m_1 = 0.5 & m_2 = 30 & m_3 = 1 & c_1 = 0.5 \\
\hline
c_2 = 10 & c_3 = 1 & l_1 = 0.5 & l_2 = 10 \\
\hline
l_3 = 1 & l_4 = 0.1 & l_5 = 0.1 & l_6 = 0.01 \\
\hline
\end{array}
\hypertarget{table_sim}
\]
\begin{center}
Таблица А.1 --- Параметры симулятора
\end{center}

\subsection*{\begin{center}Приложение В. Информация о количестве данных о валютных парах\end{center}}
\[
\begin{array}{|c|c|c|c|c|}
\hline
ETHUSDT & SOLUSDT & BNBUSDT & XRPUSDT & DOGEUSDT \\ 
\hline
326 & 78 & 100 & 53 & 169 \\
1.23\% & 0.3\% & 0.33\% & 0.14\% & 0.47\% \\
\hline
ALTUSDT & GUSDT & BANANAUSDT & LRCUSDT & HMSTRUSDT \\ 
\hline
75 & 24 & 17 & 11 & 33 \\
0.59\% & 0.68\% & 0.4\% & 0.34\% & 0.25\% \\
\hline
ATMUSDT & QUICKUSDT & EURUSDT & STRAXUSDT & BTTCUSDT \\ 
\hline
5 & 15 & 46 & 7 & 1 \\
0.18\% & 0.27\% & 0.94\% & 0.21\% & 0.93\% \\
\hline
\end{array}
\hypertarget{table_real}
\]
\begin{center}
Таблица B.1 --- Количество измерений, содержащих не менее четырех трейдов, и их доля от общего числа измерений за одну неделю
\end{center}


\end{document}