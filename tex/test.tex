\documentclass[12pt, a4paper]{article}
\usepackage[english, russian]{babel}
\usepackage[T2A]{fontenc}
\usepackage[utf8]{inputenc}
\usepackage[left=1.5cm,right=1.5cm,top=2cm,bottom=2cm,bindingoffset=0mm]{geometry}
\usepackage{titlesec}
\usepackage{amsmath}
\usepackage{amssymb}
\usepackage{setspace}
\usepackage{dsfont}
\usepackage{comment}

\setlength{\parindent}{0mm}
\linespread{1.4}

\begin{document}

\quad Рассмотрим неоднородную систему обыкновенных дифференциальных уравнений вида:
\[
\dot{x}(t) = A x(t) + \gamma(t), \eqno(1)
\]
где $x(t), \gamma(t) \in \mathds{R}^n$ - вектор-функции, $A$ - стационарная матрица размерности $n \times n$. Решение системы $(1)$ можно выписать в явном виде через формулу Коши

\[
x(t) = Y(t)\left(x(0) + \int_0^t Y^{-1}(s) \gamma(s) ds\right),
\]
где $Y(t)$ - фундаментальная матрица системы $(1)$, которую можно представить в виде:

\[
Y(t) = e^{At}.
\]

\quad Представим матрицу $A$ как $A = SDS^{-1}$, где $S$ - ортонормированная матрица, $D$ - матрица, состоящая из клеток Жордана $J_{\lambda_{i}}$. Н.У.О. рассмотрим случай, когда $D = J_{\lambda_{i}}$ размерности $n_i \times n_i$ (считаем размерность системы $n_i$), тогда $Y(t)$ представима в виде:
\[
Y(t) = S \left(
\begin{array}{ccccc}
e^{\lambda_{i}t} & te^{\lambda_{i}t} & \dots & \frac{t^{n_i - 2}}{(n_i - 2)!} e^{\lambda_{i}t} & \frac{t^{n_i - 1}}{(n_i - 1)!} e^{\lambda_{i}t} \\
0 & e^{\lambda_{i}t} & \dots & \frac{t^{n_i - 3}}{(n_i - 3)!} e^{\lambda_{i}t} & \frac{t^{n_i - 2}}{(n_i - 2)!} e^{\lambda_{i}t} \\
\vdots & \vdots & \ddots & \vdots & \vdots\\
0 & 0 & \dots & e^{\lambda_{i}t} & te^{\lambda_{i}t} \\
0 & 0 & \dots & 0 & e^{\lambda_{i}t} \\
\end{array}
\right) S^{-1},
\]
следовательно, $Y(t)Y^{-1}(s)$ представима в виде:

\[
Y(t)Y^{-1}(s) = e^{At}e^{-As} = e^{A(t-s)} = 
\] 
\[ =S \left(
\begin{array}{ccccc}
e^{\lambda_{i}(t-s)} & (t-s)e^{\lambda_{i}(t-s)} & \dots & \frac{(t-s)^{n_i - 2}}{(n_i - 2)!} e^{\lambda_{i}(t-s)} & \frac{(t-s)^{n_i - 1}}{(n_i - 1)!} e^{\lambda_{i}(t-s)} \\
0 & e^{\lambda_{i}(t-s)} & \dots & \frac{(t-s)^{n_i - 3}}{(n_i - 3)!} e^{\lambda_{i}(t-s)} & \frac{(t-s)^{n_i - 2}}{(n_i - 2)!} e^{\lambda_{i}(t-s)} \\
\vdots & \vdots & \ddots & \vdots & \vdots\\
0 & 0 & \dots & e^{\lambda_{i}(t-s)} & (t-s)e^{\lambda_{i}(t-s)} \\
0 & 0 & \dots & 0 & e^{\lambda_{i}(t-s)} \\
\end{array}
\right) S^{-1}.
\]

\newpage

\quad Сделаем замену $S^{-1}\gamma(t) = \widetilde{\gamma}(t) = (\widetilde{\gamma}_0(t), \dots, \widetilde{\gamma}_{n_i - 1}(t))^T$. Получим представление для  $Y(t)Y^{-1}(s)\gamma(s)$:
\[
Y(t)Y^{-1}(s)\gamma(s) = S \left(
\begin{array}{c}
\displaystyle \sum_{k=0}^{n_i - 1}\frac{(t-s)^k}{k!}e^{\lambda_{i}(t-s)}\widetilde{\gamma}_k(s) \\
\displaystyle \sum_{k=1}^{n_i - 1}\frac{(t-s)^k}{k!}e^{\lambda_{i}(t-s)}\widetilde{\gamma}_k(s) \\
\vdots \\
\displaystyle \sum_{k=n_i-2}^{n_i - 1}\frac{(t-s)^k}{k!}e^{\lambda_{i}(t-s)}\widetilde{\gamma}_k(s) \\
\displaystyle \sum_{k=n_i-1}^{n_i - 1}\frac{(t-s)^k}{k!}e^{\lambda_{i}(t-s)}\widetilde{\gamma}_k(s)
\end{array}
\right).
\]

\quad По итогу, выражение $x(t)$ представляем как
\[
x(t) = Y(t)x(0) + S
\left(
\begin{array}{c}
\displaystyle \sum_{k=0}^{n_i - 1}\int_0^t\frac{(t-s)^k}{k!}e^{\lambda_{i}(t-s)}\widetilde{\gamma}_k(s)ds \\
\displaystyle \sum_{k=1}^{n_i - 1}\int_0^t\frac{(t-s)^k}{k!}e^{\lambda_{i}(t-s)}\widetilde{\gamma}_k(s)ds \\
\vdots \\
\displaystyle \sum_{k=n_i-2}^{n_i - 1}\int_0^t\frac{(t-s)^k}{k!}e^{\lambda_{i}(t-s)}\widetilde{\gamma}_k(s)ds \\
\displaystyle \sum_{k=n_i-1}^{n_i - 1}\int_0^t\frac{(t-s)^k}{k!}e^{\lambda_{i}(t-s)}\widetilde{\gamma}_k(s)ds
\end{array}
\right)
.
\]

Докажем вспомогательные утверждения.

\quad \textbf{Предложение 1.} \textit{Пусть функция $f(t) \in C^1([0, \infty)) : \displaystyle\lim_{t \rightarrow \infty}\dot{f}(t) = \pm \infty \Rightarrow \lim_{t \rightarrow \infty}f(t) = \pm \infty$.}

\begin{center}
\textbf{Доказательство}
\end{center}

\quad Для определенности рассмотрим случай $\displaystyle\lim_{t \rightarrow \infty}\dot{f}(t) = +\infty$, второй случай доказывается аналогично. Из определения предела следует, что $\exists t_0 : \forall \: \widetilde{t} > t_0 \Rightarrow \dot{f}(\widetilde{t}) > 1$, то есть \\ $\displaystyle\lim_{\Delta t \rightarrow 0}\frac{f(\widetilde{t} + \Delta t) - f(\widetilde{t})}{\Delta t} > 1 \Rightarrow \exists \Delta \widetilde{t}(\widetilde{t}) > 0 : \frac{f(\widetilde{t} + \Delta \widetilde{t}) - f(\widetilde{t})}{\Delta \widetilde{t}} > 1 \Rightarrow f(\widetilde{t} + \Delta \widetilde{t}) > f(\widetilde{t}) + \Delta \widetilde{t} \Rightarrow f(\widetilde{t} + n\Delta \widetilde{t}) > f(\widetilde{t}) + n\Delta \widetilde{t}, \;\; 	\forall \: n \in \mathds{N} \Rightarrow \lim_{t \rightarrow \infty}f(t) = \lim_{n \rightarrow \infty} f(\widetilde{t} + n\Delta \widetilde{t}) \geq f(\widetilde{t}) + \Delta\widetilde{t} \lim_{n \rightarrow \infty}n = +\infty \Rightarrow \lim_{t \rightarrow \infty}f(t) = +\infty$.
\begin{flushright} $\square$ \end{flushright}

\quad \textbf{Предложение 2.} \textit{Пусть функция $f(t) \in C^1([0, \infty)) : \displaystyle\lim_{t \rightarrow \infty}f(t) = 0. \\ \exists t_0 : \forall \: t > t_0 \;\; f(t)$ - монотонна $\Rightarrow \displaystyle\lim_{t \rightarrow \infty}\dot{f}(t) = 0.$}

\begin{center}
\textbf{Доказательство}
\end{center}

\quad Для определенности рассмотрим случай, когда $\exists t_0 : \forall \: t > t_0 \;\; f(t)$ - монотоннo убывает, то есть $\exists t_0 : \forall \: \widetilde{t} > t_0 \Rightarrow \dot{f}(\widetilde{t}) < 0$. Предположим, что $\displaystyle \exists \delta > 0 : \forall t_1>t_0 \;\; \exists \widetilde{t} > t_1 : \dot{f}(\widetilde{t}) < -\delta \Rightarrow \\ \forall \: n \in \mathds{N} \lim_{\Delta t \rightarrow 0}\frac{f(\widetilde{t} + n\Delta t) - f(\widetilde{t})}{n\Delta t} <-\delta \Rightarrow \exists\Delta \widetilde{t}(n, \widetilde{t}) > 0 : f(\widetilde{t} + n\Delta t) < f(\widetilde{t}) - n\delta\Delta\widetilde{t} \Rightarrow \\ \lim_{t \rightarrow \infty}f(t) = \lim_{n \rightarrow \infty} f(\widetilde{t} + n\Delta \widetilde{t}) < f(\widetilde{t}) - \lim_{n \rightarrow \infty}n\Delta \widetilde{t} = +\infty$


\quad \textbf{Лемма 1.} \textit{Пусть функция $f(t) \in C([0, \infty))$ удовлетворяет условию $\displaystyle\lim_{t \rightarrow \infty}f(t) = \pm \infty$, то $\displaystyle\lim_{t \rightarrow \infty}\int_0^t(t-s)^k f(s)ds = \pm \infty \;\; \forall \: k \in \mathds{Z_+}$, если же $\displaystyle \exists k \in \mathds{Z_+}: \lim_{t \rightarrow \infty}\int_0^t(t-s)^k f(s)ds = 0$ и, начиная с некоторого момента, $f(s)$ сохраняет знак, то $\displaystyle\lim_{t \rightarrow \infty}f(t) = 0$.}

\quad \textbf{Доказательство.}

\quad Рассмотрим случай когда $\displaystyle\lim_{t \rightarrow \infty}f(t) = \pm \infty$. Воспользуемся следующим фактом: 
\[\forall \: g(t) \in C^1([0, \infty)) : \displaystyle\lim_{t \rightarrow \infty}\dot{g}(t) = \pm \infty \Rightarrow \lim_{t \rightarrow \infty}g(t) = \pm \infty. \eqno(2)\] 
Если в качестве $g(t)$ взять $\displaystyle\int_0^t f(s)ds$, то в силу $(2)$ получим, что $\displaystyle\lim_{t \rightarrow \infty}\int_0^tf(s)ds = \pm \infty$. Теперь возьмем в качестве $g(t) = \displaystyle\int_0^t (t-s)^kf(s)ds$, где $k \in \mathds{N}$ и предположим, что \\ $\displaystyle\lim_{t \rightarrow \infty}\int_0^t(t-s)^{k-1}f(s)ds = \pm \infty$, тогда из утверждения $(2)$ следует, что $\displaystyle\lim_{t \rightarrow \infty}\int_0^t(t-s)^kf(s)ds = \pm \infty$, тем самым мы доказали базу и шаг математической индукции.

\quad Рассмотрим второй случай леммы. Докажем следующий факт:






\begin{comment}
\quad Пусть система $(1)$ асимптотически устойчива по Ляпунову, из чего следует, что $\forall \: \lambda_{i}, \; i=\overline{1, n}$ - собственного числа матрицы $A$, выполняется $Re(\lambda_{i}) < 0$.
\end{comment} 




\end{document}