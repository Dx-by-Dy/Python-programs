\documentclass[12pt, a4paper]{article}
\usepackage[english, russian]{babel}
\usepackage[T2A]{fontenc}
\usepackage[utf8]{inputenc}
\usepackage[left=1.5cm,right=1.5cm,top=2cm,bottom=2cm,bindingoffset=0mm]{geometry}
\usepackage{titlesec}
\usepackage{amsmath}
\usepackage{amssymb}
\usepackage{setspace}
\usepackage{dsfont}
\usepackage{comment}

\setlength{\parindent}{0mm}
\linespread{1.4}

\begin{document}

\quad Рассмотрим неоднородную систему обыкновенных дифференциальных уравнений вида:
\[
\dot{x}(t) = A x(t) + \gamma(t), \eqno(1)
\]
где $x(t), \gamma(t) \in \mathds{R}^n$ - вектор-функции, $A$ - стационарная матрица размерности $n \times n$. Решение системы $(1)$ можно выписать в явном виде через формулу Коши

\[
x(t) = Y(t)\left(x(0) + \int_0^t Y^{-1}(s) \gamma(s) ds\right),
\]
где $Y(t)$ - фундаментальная матрица системы $(1)$, которую можно представить в виде:

\[
Y(t) = e^{At}.
\]

\quad Представим матрицу $A$ как $A = SDS^{-1}$, где $S$ - ортонормированная матрица, $D$ - матрица, состоящая из клеток Жордана $J_{\lambda_{i}}$. Н.У.О. рассмотрим случай, когда $D = J_{\lambda_{i}}$ размерности $n_i \times n_i$ (считаем размерность системы $n_i$), тогда $Y(t)$ представима в виде:
\[
Y(t) = S \left(
\begin{array}{ccccc}
e^{\lambda_{i}t} & te^{\lambda_{i}t} & \dots & \frac{t^{n_i - 2}}{(n_i - 2)!} e^{\lambda_{i}t} & \frac{t^{n_i - 1}}{(n_i - 1)!} e^{\lambda_{i}t} \\
0 & e^{\lambda_{i}t} & \dots & \frac{t^{n_i - 3}}{(n_i - 3)!} e^{\lambda_{i}t} & \frac{t^{n_i - 2}}{(n_i - 2)!} e^{\lambda_{i}t} \\
\vdots & \vdots & \ddots & \vdots & \vdots\\
0 & 0 & \dots & e^{\lambda_{i}t} & te^{\lambda_{i}t} \\
0 & 0 & \dots & 0 & e^{\lambda_{i}t} \\
\end{array}
\right) S^{-1},
\]
следовательно, $Y(t)Y^{-1}(s)$ представима в виде:

\[
Y(t)Y^{-1}(s) = e^{At}e^{-As} = e^{A(t-s)} = 
\] 
\[ =S \left(
\begin{array}{ccccc}
e^{\lambda_{i}(t-s)} & (t-s)e^{\lambda_{i}(t-s)} & \dots & \frac{(t-s)^{n_i - 2}}{(n_i - 2)!} e^{\lambda_{i}(t-s)} & \frac{(t-s)^{n_i - 1}}{(n_i - 1)!} e^{\lambda_{i}(t-s)} \\
0 & e^{\lambda_{i}(t-s)} & \dots & \frac{(t-s)^{n_i - 3}}{(n_i - 3)!} e^{\lambda_{i}(t-s)} & \frac{(t-s)^{n_i - 2}}{(n_i - 2)!} e^{\lambda_{i}(t-s)} \\
\vdots & \vdots & \ddots & \vdots & \vdots\\
0 & 0 & \dots & e^{\lambda_{i}(t-s)} & (t-s)e^{\lambda_{i}(t-s)} \\
0 & 0 & \dots & 0 & e^{\lambda_{i}(t-s)} \\
\end{array}
\right) S^{-1}.
\]

\newpage

\quad Сделаем замену $S^{-1}\gamma(t) = \widetilde{\gamma}(t) = (\widetilde{\gamma}_0(t), \dots, \widetilde{\gamma}_{n_i - 1}(t))^T$. Получим представление для  $Y(t)Y^{-1}(s)\gamma(s)$:
\[
Y(t)Y^{-1}(s)\gamma(s) = S \left(
\begin{array}{c}
\displaystyle \sum_{k=0}^{n_i - 1}\frac{(t-s)^k}{k!}e^{\lambda_{i}(t-s)}\widetilde{\gamma}_k(s) \\
\displaystyle \sum_{k=1}^{n_i - 1}\frac{(t-s)^{k-1}}{(k-1)!}e^{\lambda_{i}(t-s)}\widetilde{\gamma}_{k}(s) \\
\vdots \\
\displaystyle \sum_{k=n_i-2}^{n_i - 1}\frac{(t-s)^{k-(n_i-2)}}{(k-(n_i-2))!}e^{\lambda_{i}(t-s)}\widetilde{\gamma}_k(s) \\
\displaystyle \sum_{k=n_i-1}^{n_i - 1}\frac{(t-s)^{k-(n_i-1)}}{(k-(n_i-1))!}e^{\lambda_{i}(t-s)}\widetilde{\gamma}_k(s)
\end{array}
\right).
\]

\quad По итогу, выражение $x(t)$ представляем как
\[
x(t) = Y(t)x(0) + S
\left(
\begin{array}{c}
\displaystyle \sum_{k=0}^{n_i - 1}\int_0^t\frac{(t-s)^k}{k!}e^{\lambda_{i}(t-s)}\widetilde{\gamma}_k(s)ds \\
\displaystyle \sum_{k=1}^{n_i - 1}\int_0^t\frac{(t-s)^{k-1}}{(k-1)!}e^{\lambda_{i}(t-s)}\widetilde{\gamma}_{k}(s)ds \\
\vdots \\
\displaystyle \sum_{k=n_i-2}^{n_i - 1}\int_0^t\frac{(t-s)^{k-(n_i-2)}}{(k-(n_i-2))!}e^{\lambda_{i}(t-s)}\widetilde{\gamma}_k(s)ds \\
\displaystyle \sum_{k=n_i-1}^{n_i - 1}\int_0^t\frac{(t-s)^{k-(n_i-1)}}{(k-(n_i-1))!}e^{\lambda_{i}(t-s)}\widetilde{\gamma}_k(s)ds
\end{array}
\right)
. \eqno(2)
\]

Докажем вспомогательные утверждения.

\quad \textbf{Предложение 1.} \textit{Пусть функция $f(t) \in C^1([0, \infty)) : \displaystyle\lim_{t \rightarrow \infty}\dot{f}(t) = \pm \infty \Rightarrow \lim_{t \rightarrow \infty}f(t) = \pm \infty$.}

\begin{center}
\textbf{Доказательство}
\end{center}

\quad Для определенности рассмотрим случай $\displaystyle\lim_{t \rightarrow \infty}\dot{f}(t) = +\infty$, второй случай доказывается аналогично. Из определения предела следует, что $\exists t_0 : \forall \: \widetilde{t} > t_0 \Rightarrow \dot{f}(\widetilde{t}) > 1$, то есть \\ $\displaystyle\lim_{\Delta t \rightarrow 0}\frac{f(\widetilde{t} + \Delta t) - f(\widetilde{t})}{\Delta t} > 1 \Rightarrow \exists \Delta \widetilde{t}(\widetilde{t}) > 0 : \frac{f(\widetilde{t} + \Delta \widetilde{t}) - f(\widetilde{t})}{\Delta \widetilde{t}} > 1 \Rightarrow f(\widetilde{t} + \Delta \widetilde{t}) > f(\widetilde{t}) + \Delta \widetilde{t} \Rightarrow f(\widetilde{t} + n\Delta \widetilde{t}) > f(\widetilde{t}) + n\Delta \widetilde{t}, \;\; 	\forall \: n \in \mathds{N} \Rightarrow \lim_{t \rightarrow \infty}f(t) = \lim_{n \rightarrow \infty} f(\widetilde{t} + n\Delta \widetilde{t}) \geq f(\widetilde{t}) + \Delta\widetilde{t} \lim_{n \rightarrow \infty}n = +\infty \Rightarrow \lim_{t \rightarrow \infty}f(t) = +\infty$.
\begin{flushright} $\square$ \end{flushright}

\begin{comment}
\quad \textbf{Предложение 2.} \textit{Пусть функция $f(t) \in C^1([0, \infty)) : \displaystyle\lim_{t \rightarrow \infty}f(t) = 0. \\ \exists t_0 : \forall \: t > t_0 \;\; f(t)$ - монотонна $\Rightarrow \displaystyle\lim_{t \rightarrow \infty}\dot{f}(t) = 0.$}

\begin{center}
\textbf{Доказательство}
\end{center}

\quad Для определенности рассмотрим случай, когда $\exists t_0 : \forall \: t > t_0 \;\; f(t)$ - монотоннo убывает, то есть $\exists t_0 : \forall \: \widetilde{t} > t_0 \Rightarrow \dot{f}(\widetilde{t}) < 0$. Предположим, что $\displaystyle \exists \delta > 0 : \forall t_1>t_0 \;\; \exists \widetilde{t} > t_1 : \dot{f}(\widetilde{t}) < -\delta \Rightarrow \\ \forall \: n \in \mathds{N} \lim_{\Delta t \rightarrow 0}\frac{f(\widetilde{t} + n\Delta t) - f(\widetilde{t})}{n\Delta t} <-\delta \Rightarrow \exists\Delta \widetilde{t}(n, \widetilde{t}) > 0 : f(\widetilde{t} + n\Delta t) < f(\widetilde{t}) - n\delta\Delta\widetilde{t} \Rightarrow \\ \lim_{t \rightarrow \infty}f(t) = \lim_{n \rightarrow \infty} f(\widetilde{t} + n\Delta \widetilde{t}) < f(\widetilde{t}) - \lim_{n \rightarrow \infty}n\Delta \widetilde{t} = +\infty$

\end{comment}
\quad \textbf{Следствие.} \textit{Пусть функция $f(t) \in C([0, \infty))$ удовлетворяет условию $\displaystyle\lim_{t \rightarrow \infty}f(t) = \pm \infty$, то $\displaystyle\lim_{t \rightarrow \infty}\int_0^t(t-s)^k f(s)ds = \pm \infty \;\; \forall \: k \in \mathds{Z_+}$.}

\begin{comment}
если же $\displaystyle \exists k \in \mathds{Z_+}: \lim_{t \rightarrow \infty}\int_0^t(t-s)^k f(s)ds = 0$ и, начиная с некоторого момента, $f(s)$ сохраняет знак, то $\displaystyle\lim_{t \rightarrow \infty}f(t) = 0$.}
\end{comment}

\begin{center}
\textbf{Доказательство}
\end{center}

\quad По предложению 1. из условия $\displaystyle\lim_{t \rightarrow \infty}\int_0^t(t-s)^{k-1} f(s)ds = \pm \infty$ следует, что \\ $\displaystyle\lim_{t \rightarrow \infty}\int_0^t(t-s)^k f(s)ds = \pm \infty \;\; \forall \: k \in \mathds{N}$, и для $k = 0$ из $\displaystyle\lim_{t \rightarrow \infty}f(t) = \pm \infty$ следует $\displaystyle\lim_{t \rightarrow \infty}\int_0^t f(s)ds = \pm \infty$, тем самым доказаны шаг и база индукции.
\begin{flushright} $\square$ \end{flushright}

\quad \textbf{Лемма 1.} \textit{Пусть функция $f(t) \in C^1([0, \infty)) : \displaystyle\lim_{t \rightarrow \infty}\frac{\dot{f}(t)}{f(t)} = C \in \mathds{R}, C \neq \lambda$ и выполнено одно из следующих условий:}

\[
\begin{array}{cl}
1)& \displaystyle \lim_{t \rightarrow \infty}e^{-\lambda t}f(t) = \pm \infty. \\
2)& \left\{\begin{array}{l} \displaystyle\lim_{t \rightarrow \infty}\int_0^t\frac{(t-s)^m}{m!} e^{-\lambda s}f(s)ds = 0, \:\: \forall m = \overline{0,k} \\ 
\displaystyle\lim_{t \rightarrow \infty}e^{-\lambda t}f(t) = 0
\end{array}. \right.
\end{array}, \textit{то}
\]

\[\displaystyle\lim_{t \rightarrow \infty} \frac{\displaystyle\int_0^t\frac{(t-s)^k}{k!} e^{\lambda (t-s)}f(s)ds}{f(t)} = \frac{1}{(C-\lambda)^{k+1}}\]

\begin{center}
\textbf{Доказательство}
\end{center}

\quad Докажем утверждение по методу математической индукции, рассмотрим случай когда\\ $m = 0$:
$\displaystyle\lim_{t \rightarrow \infty} \frac{\displaystyle\int_0^t e^{\lambda (t-s)}f(s)ds}{f(t)} = \lim_{t \rightarrow \infty} \frac{\displaystyle\int_0^t e^{-\lambda s}f(s)ds}{e^{-\lambda t}f(t)} \triangleq $, если выполнено условие $2)$, то можно сразу применить правило Лапиталя, если выполнено условие $1)$, то, воспользовавшись следствием предложения 1., приходим к тому же выводу, следовательно
$\triangleq \displaystyle \lim_{t \rightarrow \infty} \frac{e^{-\lambda t}f(t)}{-\lambda e^{-\lambda t}f(t) + e^{-\lambda t}\dot{f}(t)} = \lim_{t \rightarrow \infty} \frac{1}{-\lambda + \frac{\dot{f}(t)}{f(t)}} = \frac{1}{C-\lambda}$. База индукции доказана. 

\quad Пусть $\displaystyle\lim_{t \rightarrow \infty} \frac{\displaystyle\int_0^t\frac{(t-s)^{m-1}}{(m-1)!} e^{\lambda (t-s)}f(s)ds}{f(t)} = \frac{1}{(C-\lambda)^m}$, тогда $\displaystyle\lim_{t \rightarrow \infty} \frac{\displaystyle\int_0^t\frac{(t-s)^{m}}{m!} e^{\lambda (t-s)}f(s)ds}{f(t)} = \\ = \lim_{t \rightarrow \infty} \frac{\displaystyle\int_0^t\frac{(t-s)^{m}}{m!} e^{-\lambda s}f(s)ds}{e^{-\lambda t}f(t)} \triangleq$, по тем же причинам можем применить правило Лапиталя и для данного предела $\triangleq \displaystyle\lim_{t \rightarrow \infty} \frac{\displaystyle\int_0^t\frac{(t-s)^{m-1}}{(m-1)!} e^{-\lambda s}f(s)ds}{-\lambda e^{-\lambda t}f(t) + e^{-\lambda t}\dot{f}(t)} = \lim_{t \rightarrow \infty} \frac{\displaystyle\int_0^t\frac{(t-s)^{m-1}}{(m-1)!} e^{-\lambda s}f(s)ds}{e^{-\lambda t}f(t)} \cdot \frac{e^{-\lambda t}f(t)}{-\lambda e^{-\lambda t}f(t) + e^{-\lambda t}\dot{f}(t)} = \\ = \lim_{t \rightarrow \infty} \frac{\displaystyle\int_0^t\frac{(t-s)^{m-1}}{(m-1)!} e^{\lambda (t-s)}f(s)ds}{f(t)}\cdot \frac{1}{-\lambda + \frac{\dot{f}(t)}{f(t)}} = \frac{1}{(C-\lambda)^m} \cdot \frac{1}{C-\lambda} = \frac{1}{(C-\lambda)^{m+1}}$. Шаг индукции доказан, а вместе с ним и лемма.
\begin{flushright} $\square$ \end{flushright}

\newpage

\quad \textbf{Следствие.}  \textit{Если $f(t)$ удовлетворяет условиям леммы, то}
\[\displaystyle\int_0^t\frac{(t-s)^k}{k!} e^{\lambda (t-s)}f(s)ds = \frac{f(t)}{(C-\lambda)^{k+1}} + o(f(t)).\]

\quad Вернемся к представлению $(2)$, получим, что если все функции $\widetilde{\gamma}_k(t), k = \overline{1,n_i}$ удовлетворяют условиям леммы $\left(\displaystyle\lim_{t \rightarrow \infty}\frac{\dot{\widetilde{\gamma}}_k(t)}{\widetilde{\gamma}_k(t)} = C_k \neq \lambda_i \:\: \forall k=\overline{0, n_i-1}\right)$, то $x(t)$ можно представить как 

\[
x(t) = Y(t)x(0) + S
\left(
\begin{array}{c}
\displaystyle \sum_{k=0}^{n_i - 1}\left(\frac{\widetilde{\gamma}_k(t)}{(C_k-\lambda_i)^{k+1}} + o(\widetilde{\gamma}_k(t))\right) \\
\displaystyle \sum_{k=1}^{n_i - 1}\left(\frac{\widetilde{\gamma}_k(t)}{(C_k-\lambda_i)^{k}} + o(\widetilde{\gamma}_k(t))\right) \\
\vdots \\
\displaystyle \sum_{k=n_i-2}^{n_i - 1}\left(\frac{\widetilde{\gamma}_k(t)}{(C_k-\lambda_i)^{k+1-(n_i-2)}} + o(\widetilde{\gamma}_k(t))\right) \\
\displaystyle \sum_{k=n_i-1}^{n_i - 1}\left(\frac{\widetilde{\gamma}_k(t)}{(C_k-\lambda_i)^{k+1-(n_i-1)}} + o(\widetilde{\gamma}_k(t))\right)
\end{array}
\right)
.
\]

\quad Все $o(\widetilde{\gamma}_k(t))$ можно заменить на обозначение вектора о-малое , то есть просто обозначим 
\[o(\widetilde{\gamma}(t)) = 
\left(
\begin{array}{c}
\displaystyle \sum_{k=0}^{n_i - 1}o(\widetilde{\gamma}_k(t)) \\
\displaystyle \sum_{k=1}^{n_i - 1}o(\widetilde{\gamma}_k(t)) \\
\vdots \\
\displaystyle \sum_{k=n_i-2}^{n_i - 1}o(\widetilde{\gamma}_k(t)) \\
\displaystyle \sum_{k=n_i-1}^{n_i - 1}o(\widetilde{\gamma}_k(t))
\end{array}
\right),
\]

также введем матрицу $P_i^{\widetilde{\gamma}}$
\[ P_i^{\widetilde{\gamma}} = \left(
\begin{array}{ccccc}
\displaystyle\frac{1}{C_0-\lambda_i} & \displaystyle\frac{1}{(C_1-\lambda_i)^2} & \dots & \displaystyle\frac{1}{(C_{n_i-2}-\lambda_i)^{n_i-1}} & \displaystyle\frac{1}{(C_{n_i-1}-\lambda_i)^{n_i}} \\
0 & \displaystyle\frac{1}{C_1-\lambda_i} & \dots & \displaystyle\frac{1}{(C_{n_i-2}-\lambda_i)^{n_i-2}} & \displaystyle\frac{1}{(C_{n_i-1}-\lambda_i)^{n_i-1}} \\
\vdots & \vdots & \ddots & \vdots & \vdots\\
0 & 0 & \dots & \displaystyle\frac{1}{C_{n_i-2}-\lambda_i} & \displaystyle\frac{1}{(C_{n_i-1}-\lambda_i)^2} \\
0 & 0 & \dots & 0 & \displaystyle\frac{1}{C_{n_i-1}-\lambda_i} \\
\end{array}
\right).
\]
\newpage

\quad По итогу $x(t)$ можно представить как
\[
x(t) = Y(t)x(0) + S\left(P_i^{\widetilde{\gamma}} \: \widetilde{\gamma}(t) + o(\widetilde{\gamma}(t))\right) = Y(t)x(0) + S P_i^{\widetilde{\gamma}} S^{-1} \gamma(t) + S \left( o(\widetilde{\gamma}(t))\right).
\]

\quad Индекс $i$ напоминает нам, что мы работали только с одной Жордановой клеткой матрицы $D$, обобщая наши рассуждения на полный вид матрицы $D$, матрица $P_i^{\widetilde{\gamma}}$ заменится на матрицу $P^{\widetilde{\gamma}}$, у которой по диагонали стоят соответствующие матрицы $P_i^{\widetilde{\gamma}}$, то есть получим, что
\[
x(t) = Y(t)x(0) + S P^{\widetilde{\gamma}} S^{-1} \gamma(t) + S \left( o(\widetilde{\gamma}(t))\right). \eqno(3)
\]

\quad Подставим представление $(3)$ в систему $(1)$, получим
\[
\begin{array}{c}
AY(t)x(0) + S P^{\widetilde{\gamma}} S^{-1} \dot{\gamma}(t) + S \left( \dot{o}(\widetilde{\gamma}(t))\right) = AY(t)x(0) + A S P^{\widetilde{\gamma}} S^{-1} \gamma(t) + AS\left( o(\widetilde{\gamma}(t))\right) + \gamma(t)
\\
S \left( \dot{o}(\widetilde{\gamma}(t))\right) = SDP^{\widetilde{\gamma}} S^{-1} \gamma(t) - S P^{\widetilde{\gamma}} S^{-1} \dot{\gamma}(t) + SD\left( o(\widetilde{\gamma}(t))\right) + \gamma(t)
\\
\dot{o}(\widetilde{\gamma}(t)) = DP^{\widetilde{\gamma}} S^{-1} \gamma(t) - P^{\widetilde{\gamma}} S^{-1} \dot{\gamma}(t) + D\left( o(\widetilde{\gamma}(t))\right) + S^{-1} \gamma(t)
\\
\dot{o}(\widetilde{\gamma}(t)) = D\left( o(\widetilde{\gamma}(t))\right) + \left(DP^{\widetilde{\gamma}} + \mathds{E}\right) S^{-1} \gamma(t) - P^{\widetilde{\gamma}} S^{-1} \dot{\gamma}(t).
\end{array}
\]

\quad Рассмотрим $\left(DP^{\widetilde{\gamma}} + \mathds{E}\right)$, Н.У.О. рассмотрим произведение $J_{\lambda_i} P_i^{\widetilde{\gamma}}$
\[
\left(
\begin{array}{ccccc}
\lambda_i & 1 & \dots & 0 & 0 \\
0 & \lambda_i & \dots & 0 & 0 \\
\vdots & \vdots & \ddots & \vdots & \vdots\\
0 & 0 & \dots & \lambda_i & 1 \\
0 & 0 & \dots & 0 & \lambda_i \\
\end{array}
\right)
\left(
\begin{array}{ccccc}
\displaystyle\frac{1}{C_0-\lambda_i} & \displaystyle\frac{1}{(C_1-\lambda_i)^2} & \dots & \displaystyle\frac{1}{(C_{n_i-2}-\lambda_i)^{n_i-1}} & \displaystyle\frac{1}{(C_{n_i-1}-\lambda_i)^{n_i}} \\
0 & \displaystyle\frac{1}{C_1-\lambda_i} & \dots & \displaystyle\frac{1}{(C_{n_i-2}-\lambda_i)^{n_i-2}} & \displaystyle\frac{1}{(C_{n_i-1}-\lambda_i)^{n_i-1}} \\
\vdots & \vdots & \ddots & \vdots & \vdots\\
0 & 0 & \dots & \displaystyle\frac{1}{C_{n_i-2}-\lambda_i} & \displaystyle\frac{1}{(C_{n_i-1}-\lambda_i)^2} \\
0 & 0 & \dots & 0 & \displaystyle\frac{1}{C_{n_i-1}-\lambda_i} \\
\end{array}
\right) = 
\]
\[
= \left(
\begin{array}{ccccc}
\displaystyle\frac{\lambda_i}{C_0-\lambda_i} & \displaystyle\frac{C_1}{(C_1-\lambda_i)^2} & \dots & \displaystyle\frac{C_{n_i-2}}{(C_{n_i-2}-\lambda_i)^{n_i-1}} & \displaystyle\frac{C_{n_i-1}}{(C_{n_i-1}-\lambda_i)^{n_i}} \\
0 & \displaystyle\frac{\lambda_i}{C_1-\lambda_i} & \dots & \displaystyle\frac{C_{n_i-2}}{(C_{n_i-2}-\lambda_i)^{n_i-2}} & \displaystyle\frac{C_{n_i-1}}{(C_{n_i-1}-\lambda_i)^{n_i-1}} \\
\vdots & \vdots & \ddots & \vdots & \vdots\\
0 & 0 & \dots & \displaystyle\frac{\lambda_i}{C_{n_i-2}-\lambda_i} & \displaystyle\frac{C_{n_i-1}}{(C_{n_i-1}-\lambda_i)^2} \\
0 & 0 & \dots & 0 & \displaystyle\frac{\lambda_i}{C_{n_i-1}-\lambda_i} \\
\end{array}
\right),
\]

соответственно $\left(J_{\lambda_i} P_i^{\widetilde{\gamma}} + \mathds{E}\right)$ есть
\[
\left(
\begin{array}{ccccc}
\displaystyle\frac{C_0}{C_0-\lambda_i} & \displaystyle\frac{C_1}{(C_1-\lambda_i)^2} & \dots & \displaystyle\frac{C_{n_i-2}}{(C_{n_i-2}-\lambda_i)^{n_i-1}} & \displaystyle\frac{C_{n_i-1}}{(C_{n_i-1}-\lambda_i)^{n_i}} \\
0 & \displaystyle\frac{C_1}{C_1-\lambda_i} & \dots & \displaystyle\frac{C_{n_i-2}}{(C_{n_i-2}-\lambda_i)^{n_i-2}} & \displaystyle\frac{C_{n_i-1}}{(C_{n_i-1}-\lambda_i)^{n_i-1}} \\
\vdots & \vdots & \ddots & \vdots & \vdots\\
0 & 0 & \dots & \displaystyle\frac{C_{n_i-2}}{C_{n_i-2}-\lambda_i} & \displaystyle\frac{C_{n_i-1}}{(C_{n_i-1}-\lambda_i)^2} \\
0 & 0 & \dots & 0 & \displaystyle\frac{C_{n_i-1}}{C_{n_i-1}-\lambda_i} \\
\end{array}
\right),
\]

матрица $\left(DP^{\widetilde{\gamma}} + \mathds{E}\right)$ состоит из матричных блоков на диагонали вида 
$\left(J_{\lambda_i} P_i^{\widetilde{\gamma}} + \mathds{E}\right)$, обозначим $\left(DP^{\widetilde{\gamma}} + \mathds{E}\right) = \overline{P^{\widetilde{\gamma}}}$. 

\quad \textbf{Определение 1.}

\quad Рассмотрим $\overline{P^{\widetilde{\gamma}}}\widetilde{\gamma}(t) - P^{\widetilde{\gamma}}\dot{\widetilde{\gamma}}(t) = \theta(t)$, покажем, что $\displaystyle\lim_{t \rightarrow \infty}\left|\frac{\theta_k(t)}{\displaystyle\max_j(\widetilde{\gamma}_j(t))}\right| = 0 \\ \forall k=\overline{0, n}$, где $\displaystyle\max_j(\widetilde{\gamma}_j(t))$ есть такая $\widetilde{\gamma}_{j_m}(t)$, что $\displaystyle \overline{\lim_{t \rightarrow \infty}}\left|\frac{\widetilde{\gamma}_k(t)}{\widetilde{\gamma}_{j_m}(t)}\right| \leq 1 \:\: \forall k=\overline{0, n}$.

\quad Рассмотрим $\displaystyle\lim_{t \rightarrow \infty}\left|\frac{\theta_1(t)}{\displaystyle\max_j(\widetilde{\gamma}_j(t))}\right| = \lim_{t \rightarrow \infty}\left|\frac{\displaystyle \sum_{k=0}^{n_1-1}\frac{C_k \widetilde{\gamma}_k(t) - \dot{\widetilde{\gamma}}_k(t)}{(C_k-\lambda_1)^{k+1}}}{\displaystyle\max_j(\widetilde{\gamma}_j(t))}\right| = \lim_{t \rightarrow \infty}\left| \displaystyle \sum_{k=0}^{n_1-1}\frac{\widetilde{\gamma}_k(t)}{\displaystyle\max_j(\widetilde{\gamma}_j(t))} \cdot \frac{C_k \ - \displaystyle \frac{\dot{\widetilde{\gamma}}_k(t)}{\widetilde{\gamma}_k(t)}}{(C_k-\lambda_1)^{k+1}}\right|$, получаем, что \[0 \leq \displaystyle\lim_{t \rightarrow \infty}\left|\frac{\theta_1(t)}{\displaystyle\max_j(\widetilde{\gamma}_j(t))}\right| \leq \lim_{t \rightarrow \infty} \displaystyle \sum_{k=0}^{n_1-1}\left|\frac{\widetilde{\gamma}_k(t)}{\displaystyle\max_j(\widetilde{\gamma}_j(t))} \cdot \frac{C_k \ - \displaystyle \frac{\dot{\widetilde{\gamma}}_k(t)}{\widetilde{\gamma}_k(t)}}{(C_k-\lambda_1)^{k+1}}\right| \leq \lim_{t \rightarrow \infty} \displaystyle \sum_{k=0}^{n_1-1}\left| \frac{C_k \ - \displaystyle \frac{\dot{\widetilde{\gamma}}_k(t)}{\widetilde{\gamma}_k(t)}}{(C_k-\lambda_1)^{k+1}}\right| = 0,\]

абсолютно аналогичное доказательство для любого другого $k = \overline{2, n}$, с точностью до пределов суммирования.

\quad Из того, что $\displaystyle\lim_{t \rightarrow \infty}\left|\frac{\theta_k(t)}{\displaystyle\max_j(\widetilde{\gamma}_j(t))}\right| = 0 \:\: \forall k=\overline{0, n} \Rightarrow \displaystyle\lim_{t \rightarrow \infty}\frac{\theta_k(t)}{\displaystyle\max_j(\widetilde{\gamma}_j(t))} = 0 \:\: \forall k=\overline{0, n} \Rightarrow \\ \Rightarrow \lim_{t \rightarrow \infty}\frac{\displaystyle\max_j(\theta_j(t))}{\displaystyle\max_j(\widetilde{\gamma}_j(t))} = 0 \Rightarrow$ будем говорить, что порядок $\theta(t)$




\end{document}